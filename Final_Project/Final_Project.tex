% % % % % % % % % % % % % % % % % % % % % % % % % % % % % % % % % % % % % % % %
% IEEE Style - Double columns, 11pt font, letterpaper
\documentclass[journal, twocolumn, final,11pt,letterpaper]{IEEEtran}	

% Include Latex Packages
\usepackage{etex}	% This package enables the use of many packages

% % Page styles
\usepackage{setspace}	% line spacing package
\doublespacing			% use double spacing
%\linespread{1.6}		% Use linespread to fine tune line spacing, not recommended


% % Figures
\usepackage{float}		% improves interface for floating objects
\usepackage{subfig}		% enables subfloat
\usepackage{graphicx}	% more image type support
\usepackage{epstopdf}	% automatically convert included eps files to pdf

% % Maths
\usepackage[cmex10]{amsmath}	% Maths
\usepackage{amsfonts,amssymb} 	% maths symbols

% % Tables
\usepackage{booktabs}  % professional-looking tables
\usepackage{multicol} %used for getting multicolumn without page-break
\usepackage{multirow}	% multi-row tables
\usepackage{array}		% define column format of a table

% % Others
\usepackage{caption}	%Customising captions in floating environments
%\usepackage{abstract}
\usepackage{cite}		% cite multiple
\usepackage{fixltx2e}	%added by pilawa, preventing figure* to get ahead of regular figures.
\usepackage{url}		% url display

% %
\hyphenation{op-tical net-works semi-conduc-tor}	% correct bad hyphenation here
\providecommand{\e}[1]{\ensuremath{\times 10^{#1}}}		% use use \e{2} for scientific number expression


% % Optional packages that might be useful
%\usepackage{epsf}		% eps fix
%\usepackage{verbatim}	% verbatim text are not interpreted by the compiler 
%\numberwithin{equation}{section}	% number equation according to section
%\usepackage{xfrac}		% slanted fraction
%\usepackage{pgfplots}	% plot graph
%\usepackage{tikz,pgfplots} % plot graph
%\usepackage{endnotes}	% endnotes


% Title of Document
\title{ECE385 Final Project Report
	}
\author{
\IEEEauthorblockN{Frogger in System Verilog\\ Eric Meyers, Ryan Helsdingen}\\
\IEEEauthorblockA{Section ABG; TAs: Ben Delay, Shuo Liu \\
May 4th, 2016 \\
emeyer7, helsdin2}}
% % % % % % % % % % % % % % % % % % % % % % % % % % % % % % % % % % % % % % % 
\begin{document}
	
%SECTION : Formatting and Title
\maketitle
\singlespacing

%SECTION 2 - Description of Circuit / Inputs and Outputs - 
\section{Introduction}
The basic premise of Frogger is to navigate frogs across the street/water without dying. A frog may die by either colliding with a moving car or falling in the water. There are a total of three frogs that the user must navigate to the other end of the map, and once all three frogs move to their particular ending location, the user wins. If a user dies three times, then the game is over.\\

This system was developed in System Verilog in Quartus-II, and used software drivers developed in C to communicate with a USB keyboard (to be used as the controller).
%Our idea is to create a "Frogger" game with the basic premise of moving a frog across the street without getting hit by any moving obstacles (refer to Figure \ref{fig:frogger} for more details). This will be accomplished via similar techniques used in lab8 and interfacing VGA graphics with a USB keyboard controller.\\
%
%A NIOS-II Processor will be used which will be created in Qsys along with PIO modules to handle the input and output of the game. The sprite data and the background image data will be stored onboard the SDRAM which holds 1GBit of data (~20MB) - this will be plenty enough to store the sprite data ($<$ 1MB) and any other background image data we choose to use. The color\_mapper.sv file will then be modified to include the particular sprites wherever/whenever we may choose to use them.
\section{List of Features}
%\begin{itemize}
%	\item User controlled block moves according to grid set on VGA display
%	\begin{itemize}
%		\item Up, down, left, or right depending on input
%	\end{itemize}
%	\item Moving obstacles that are different shapes
%	\begin{itemize}
%		\item Different types of objects can lead to different outcomes - i.e. object can either allow ``frogger" to move with it or kill it.
%	\end{itemize}
%	\item Multiple levels with increasing difficulty
%	\item Starting point and ending point on any given level
%	\item Timer
%		\begin{itemize}
%			\item 5 minutes to complete level
%		\end{itemize} 
%	\item Color 
%	\begin{itemize}
%		\item Must be able to clearly differentiate between obstacle, user controlled block, and the map
%	\end{itemize}
%	\item Score/Highscore
%\end{itemize}

%\vspace{5mm}
%
%\textit{Optional Functionality and Complexity}
%\begin{itemize}
%	\item Multiple maps
%	\begin{itemize}
%		\item Maps taking place with different shaped obstacles and different background
%	\end{itemize}
%	\item Sound - 8-bit soundtrack
%	\item Sprites and animations
%	\item Start menu - Options Help Highscores, Start Button
%	\item Powerups:
%		\begin{itemize}
%			\item Slow-down/speed-up obstacles
%			\item Longer blocks for "frogger" to hop onto
%			\item 
%		\end{itemize}
%	\item 2-Player Mode
%\end{itemize}
%SECTION 3 - Purpose of Modules - 
\section{Block Diagram} 
%\begin{figure} [H]
%	\centering
%	\includegraphics[scale=.25]{Block_Diagram.png}
%	\caption{Block Diagram\label{fig:frogger}}
%\end{figure} 

\section{Purpose of Modules}

\section{Circuit Schematics}

\section{Finite State Machines}
%SECTION 4 - Description of USB Protocol & Changes


%SECTION 5 - State Diagram for Decryption 
\section{Color \& Sprite Generation}

\section{ Difficulty}
%The basic functionality of this game will not be much difficulty at all (4 pts). We are relying on implementing a majority of our ``Optional Functionality and Complexity" that will give us the bulk of the difficulty points. This project will most likely approach a 6 in terms of difficulty. \\
%
%The basic functionality is not too difficult to implement and will require us to heavily modify Lab 8. Sprites, animations, different maps, levels and power-ups will all require efficient memory usage. 

%SECTION 6 - Proposed Timeline
\section{Conclusion} 
%The final design project will take a total of 5 weeks to complete. Establishing a timeline will help guide the project to completion.  The table below gives an overview of lab dates and key checkpoints for the project.
%
%\begin{table}[htbp]
%	\centering
%	\begin{tabular}{ccc}	% ccccccc indicates 7 center aligned columns
%		\toprule	% top separator
%		Week \# & Lab Date & Checkpoint \\
%		\midrule
%		0 & March 30 & \\
%		1 & April 6 & Design Proposal Due \\
%		2 & April 13 & \\
%		3 & April 20 & Mid-checkpoint \\
%		4 & April 27 & \\
%		5 & May 4 & Final Demo/Report Due \\
%		
%		\bottomrule	% bottom separator
%	\end{tabular}%
%	\label{tab:table1}	% this is the label given to the table that can be referenced using \ref{tab:Exp1Part1_7}
%\end{table}%

%\section{Conclusion}


%SECTION 7: Figures
\section{Figures}
\section{Appendix}


%\begin{figure} [H]
%	\centering
%	\includegraphics[scale=.3]{Frogger.png}
%	\caption{Basic Gameplay Demonstration\label{fig:frogger}}
%\end{figure}            

     

\end{document}
