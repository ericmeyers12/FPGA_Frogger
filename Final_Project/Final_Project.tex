% % % % % % % % % % % % % % % % % % % % % % % % % % % % % % % % % % % % % % % %
% IEEE Style - Double columns, 11pt font, letterpaper
\documentclass[journal, twocolumn, final,11pt,letterpaper]{IEEEtran}	

% Include Latex Packages
\usepackage{etex}	% This package enables the use of many packages

% % Page styles
\usepackage{setspace}	% line spacing package
\doublespacing			% use double spacing
%\linespread{1.6}		% Use linespread to fine tune line spacing, not recommended


% % Figures
\usepackage{float}		% improves interface for floating objects
\usepackage{subfig}		% enables subfloat
\usepackage{graphicx}	% more image type support
\usepackage{epstopdf}	% automatically convert included eps files to pdf

% % Maths
\usepackage[cmex10]{amsmath}	% Maths
\usepackage{amsfonts,amssymb} 	% maths symbols

% % Tables
\usepackage{booktabs}  % professional-looking tables
\usepackage{multicol} %used for getting multicolumn without page-break
\usepackage{multirow}	% multi-row tables
\usepackage{array}		% define column format of a table

% % Others
\usepackage{caption}	%Customising captions in floating environments
%\usepackage{abstract}
\usepackage{cite}		% cite multiple
\usepackage{fixltx2e}	%added by pilawa, preventing figure* to get ahead of regular figures.
\usepackage{url}		% url display

% %
\hyphenation{op-tical net-works semi-conduc-tor}	% correct bad hyphenation here
\providecommand{\e}[1]{\ensuremath{\times 10^{#1}}}		% use use \e{2} for scientific number expression


% % Optional packages that might be useful
%\usepackage{epsf}		% eps fix
%\usepackage{verbatim}	% verbatim text are not interpreted by the compiler 
%\numberwithin{equation}{section}	% number equation according to section
%\usepackage{xfrac}		% slanted fraction
%\usepackage{pgfplots}	% plot graph
%\usepackage{tikz,pgfplots} % plot graph
%\usepackage{endnotes}	% endnotes


% Title of Document
\title{ECE385 Final Project Report
	}
\author{
\IEEEauthorblockN{Frogger in System Verilog\\ Eric Meyers, Ryan Helsdingen}\\
\IEEEauthorblockA{Section ABG; TAs: Ben Delay, Shuo Liu \\
May 4th, 2016 \\
emeyer7, helsdin2}}
% % % % % % % % % % % % % % % % % % % % % % % % % % % % % % % % % % % % % % % 
\begin{document}
	
%SECTION : Formatting and Title
\maketitle
\singlespacing

%SECTION 1 - Introduction - ERIC
\section{Introduction}
The basic premise of Frogger is to navigate frogs across the street/water without dying. A frog may die by either colliding with a moving car or falling in the water. There are a total of three frogs that the user must navigate to the other end of the map, and once all three frogs move to their particular ending location, the user wins. If a user dies three times, then the game is over.\\

This system was developed in System Verilog in Quartus-II on an Altera-DE2-115 FPGA Board, and used software drivers developed in C to communicate with a USB keyboard (to be used as the controller).


%SECTION 2 - List of Features
\section{List of Features}
	%RYAN SECTION
%\begin{itemize}
%	\item User controlled block moves according to grid set on VGA display
%	\begin{itemize}
%		\item Up, down, left, or right depending on input
%	\end{itemize}
%	\item Moving obstacles that are different shapes
%	\begin{itemize}
%		\item Different types of objects can lead to different outcomes - i.e. object can either allow ``frogger" to move with it or kill it.
%	\end{itemize}
%	\item Multiple levels with increasing difficulty
%	\item Starting point and ending point on any given level
%	\item Timer
%		\begin{itemize}
%			\item 5 minutes to complete level
%		\end{itemize} 
%	\item Color 
%	\begin{itemize}
%		\item Must be able to clearly differentiate between obstacle, user controlled block, and the map
%	\end{itemize}
%	\item Score/Highscore
%\end{itemize}

%\vspace{5mm}
%
%\textit{Optional Functionality and Complexity}
%\begin{itemize}
%	\item Multiple maps
%	\begin{itemize}
%		\item Maps taking place with different shaped obstacles and different background
%	\end{itemize}
%	\item Sound - 8-bit soundtrack
%	\item Sprites and animations
%	\item Start menu - Options Help Highscores, Start Button
%	\item Powerups:
%		\begin{itemize}
%			\item Slow-down/speed-up obstacles
%			\item Longer blocks for "frogger" to hop onto
%			\item 
%		\end{itemize}
%	\item 2-Player Mode
%\end{itemize}

%SECTION 3 - Block Diagram - 
\section{Block Diagram} 
	 %ERIC SECTION - PUT BETTER BLOCK DIAGRAM HERE
	 %RYAN - if you have anything to add here, feel free too
 
 
%SECTION 4 - Purpose of Modules - 
\section{Purpose of Modules}
	%ERIC SECTION/RYAN SECTION- WORK ON YOUR RESPECTIVE STUFF
	
%SECTION 5 - Circuit Schematics - 
\section{Circuit Schematics}
	%ERIC SECTION

%SECTION 6 - Finite State Machines - 
\section{Finite State Machines}
	%RYAN SECTION?

%SECTION 7 - Color and Sprite Generation
\section{Color \& Sprite Generation}
	%RYAN SECTION?

%SECTION 8 - Difficulty - 
\section{ Difficulty}
	%RYAN SECTION - Make an argument for the difficulty of this project


%SECTION 9 - Conclusion
\section{Conclusion} 
	%JOINT SECTION



%SECTION 10: Figures & Appendix
\section{Figures}
\section*{Appendix}


%%\begin{figure} [H]
%%	\centering
%%	\includegraphics[scale=.3]{Frogger.png}
%%	\caption{Basic Gameplay Demonstration\label{fig:frogger}}
%%\end{figure}            
%
%     

\end{document}
