% % % % % % % % % % % % % % % % % % % % % % % % % % % % % % % % % % % % % % % %
% IEEE Style - Double columns, 11pt font, letterpaper
\documentclass[journal, twocolumn, final,11pt,letterpaper]{IEEEtran}	

% Include Latex Packages
\usepackage{etex}	% This package enables the use of many packages

% % Page styles
\usepackage{setspace}	% line spacing package
\doublespacing			% use double spacing
%\linespread{1.6}		% Use linespread to fine tune line spacing, not recommended


% % Figures
\usepackage{float}		% improves interface for floating objects
\usepackage{subfig}		% enables subfloat
\usepackage{graphicx}	% more image type support
\usepackage{epstopdf}	% automatically convert included eps files to pdf

% % Maths
\usepackage[cmex10]{amsmath}	% Maths
\usepackage{amsfonts,amssymb} 	% maths symbols

% % Tables
\usepackage{booktabs}  % professional-looking tables
\usepackage{multicol} %used for getting multicolumn without page-break
\usepackage{multirow}	% multi-row tables
\usepackage{array}		% define column format of a table

% % Others
\usepackage{caption}	%Customising captions in floating environments
%\usepackage{abstract}
\usepackage{cite}		% cite multiple
\usepackage{fixltx2e}	%added by pilawa, preventing figure* to get ahead of regular figures.
\usepackage{url}		% url display

% %
\hyphenation{op-tical net-works semi-conduc-tor}	% correct bad hyphenation here
\providecommand{\e}[1]{\ensuremath{\times 10^{#1}}}		% use use \e{2} for scientific number expression


% % Optional packages that might be useful
%\usepackage{epsf}		% eps fix
%\usepackage{verbatim}	% verbatim text are not interpreted by the compiler 
%\numberwithin{equation}{section}	% number equation according to section
%\usepackage{xfrac}		% slanted fraction
%\usepackage{pgfplots}	% plot graph
%\usepackage{tikz,pgfplots} % plot graph
%\usepackage{endnotes}	% endnotes


% Title of Document
\title{ECE385 Experiment \#6
	}
\author{
\IEEEauthorblockN{Eric Meyers, Ryan Helsdingen}\\
\IEEEauthorblockA{Section ABG; TAs: Ben Delay, Shuo Liu \\
March 9th, 2016 \\
emeyer7, helsdin2}}
% % % % % % % % % % % % % % % % % % % % % % % % % % % % % % % % % % % % % % % 
\begin{document}
	
%SECTION : Formatting and Title
\maketitle
\singlespacing

%SECTION 1 - Introduction - Eric
\section{Introduction}
The purpose of this lab is to create a very primitive processing unit designed around the Little Computer 3 (LC3) that was explored during previous ECE curriculum. This will be referred to the SLC3 Processing Unit throughout this lab. The SLC3 is a condensed version of the LC3 that allows user interfacing through memory-mapped I/O on board the Altera Cyclone IV SRAM Module, along with switches and LED indicators to show the user the status of the data within registers of the SLC3.  

%SECTION 2 - Description of Circuit - 
\section{Description of Circuit}
The circuit consists of several modules; specifically the high level SLC­3 module, the register file, the datapath, the Instruction Decoder/Sequencer Unit (IDSU), the Arithmetic and Logic Unit (ALU), many 16-bit registers (MAR, MDR, IR, and PC), several multiplexers, and some tristate buffers. The datapath and ISDU (Control) are shown in Figure \ref{fig:SLC3-Circuit}. This Figure directly below this in the same section (Figure \ref{fig:Memory-Circuit}) is the memory interface of the SLC3.\\

All of these modules work together with one another to form the top level SLC3. The SLC3 will perform a total of 

TODO

%SECTION 3 - Purpose of Modules - 
\section{Purpose of Modules}
As stated in the previous section there are many modules that work together in this system to form the top level SLC3. The following modules were created:\\

\normalsize\textbf{16-bit Shift Register} \\
These modules are 16 inputs and 16 outputs with a load\_enable line that determines if the data-in is sent to the data-out. The Instruction Register, Memory Address Register, Memory Data Register, Program Counter Register, and Register File will all be utilizing this module. \\

\normalsize\textbf{Multiplexers} \\
A 16-bit 2-to-1 MUX will be used for the ADDR1MUX and the SR2MUX. A 3-bit 2-to-1 MUX will be used for the DRMUX and the SR1MUX.

A 16-bit 4-to-1 MUX will be used for the ADDR2MUX only.

A 16-bit 3-to-1 MUX will be used for the PCMUX. This will take inputs of the databus, the 16-bit adder output, and the PC+1. \\

\normalsize\textbf{Instruction Sequencer and Decoder } \\
This module will contain the state machine for the entire SLC3. It will have the ability to implement 11 states and cycle through all of them in a cyclic fashion. \\

\normalsize\textbf{NZP} \\
TODO \\

\normalsize\textbf{Datapath} \\
TODO \\

\normalsize\textbf{Arthmetic and Logic Unit} \\
\begin{table}[htbp]
	\centering
	%	\caption{Short Circuit Test Measurements.}
	\begin{tabular}{cc}	% ccccccc indicates 7 center aligned columns
		\toprule	% top separator
		ALUK & OUTPUT FUNCTION \\
		\midrule
		00 & A ADD B \\
		01 & A AND B \\
		10 & NOT A \\
		11 & PASS A \\
		\bottomrule	% bottom separator
	\end{tabular}%
	\label{tab:ALU-table}	% this is the label given to the table that can be referenced using \ref{tab:Exp1Part1_7}
\end{table}%

The ALU module does the brunt arithmetic work of the SLC3 machine.  The ALU takes in two 16 values (A and B) and does one of four things with the data: bitwise ADD, bitwise AND, bitwise NOT, or passes the A input through to the output terminal. \\



\normalsize\textbf{SEXT/ZEXT} \\
The SEXT/ZEXT modules take in data that is M bits long and extend it to N bits long where M<N by adding bits to the front of the input.  ZEXT extends the input with purely zeros added to the front of it, whereas SEXT depends on the sign of the value.  If the MSB=1, the input is negative and gets extended with the proper number of 1s.  On the contrary, if MSB=0, the input is positive and gets extended with the proper number of zeros.  \\

\normalsize\textbf{Register File} \\
TODO \\

\normalsize\textbf{16-bit Adder} \\
TODO \\

\normalsize\textbf{Tristate Buffer} \\
The tristate buffer module is used to protect the bus from multiple signals entering the bus.  The tristate buffer takes in gate signals determined by the state of the machine and sends the respective data to the proper databus input. \\

\normalsize\textbf{Tristate} \\
TODO \\

\normalsize\textbf{Mem2IO} \\
TODO \\




%SECTION 4 - State Diagram - 
\section{State Diagram}
The State Diagram can be found on Figure \ref{state-diagram} in "Section XI: Figures".

%SECTION 5 - Instruction Sequencer / Decoder - 
\section{Instruction Sequencer / Decoder}
ERIC SECTION 

%SECTION 6 - Schematic/Block Diagram - 
\section{Schematic/Block Diagram}
RYAN SECTION\\

The Schematic / Block Diagrams can be found on Figure \ref{} in "Section XI: Figures".

%SECTION 7 - Pre-Lab Simulation Waveforms -
\section{Pre-Lab Simulation Waveforms}
The Pre-Lab Simulation Waveforms can be found on Figure \ref{} in "Section XI: Figures".

%SECTION 8 - Design Statistics
\section{Design Statistics}
ERIC SECTION

%SECTION 9 - Post Lab -Ryan
\section{Post Lab}
%RYAN SECTION - Answer Questions \\

1.) What is MEM2IO used for, i.e. what is its main function? \\

MEM2IO acts as the middle-man between the data going to/from the SRAM and CPU and the I/O of the system.  For this lab, the I/O includes the switches as input and the Hex displays as output. \\  

2.) What is the difference between BR and JMP instructions?\\

While the BR branch instruction and the JMP jump instruction pushes the PC to a non-sequential location as a result, they are completely different functions in design.  The JMP instruction is simple in design and function.  It takes the value in BaseR and shoves it into the PC regardless of any other conditions.  The BR instruction is conditional.  It relies on the current values of n, z, and p.  If any of the values match their equivalents in the status register, the branch is enabled and the value of PC is updated with the new address.  If none of the values match, the branch instruction is ignored and the next instruction is called.  \\    

%SECTION 10 - Conclusion - Ryan
\section{Conclusion}
Week 1 was successful for the team as we finished the Fetch command with relative ease.  The team had minor problems with understanding the datapath vs. databus, minimizing delay in data movement, and properly implementing test\_memory.sv into the design for simulation and testing.  Initially too many modules were created building a high enough delay to the point that signals were missing their proper clock cycles.  \\

Week 2 gave the team much more trouble.  Each function required some sort of debugging after the first failed attempt.  The largest source of errors was improper connections between modules.  Unfortunately, not all of the functions were able to be completely debugged in time for the lab section. \\
%expand here

Beyond the problems experienced in lab, understanding the creation of the simple SLC3 processor in System Verilog is a valuable tool for a computer engineer moving foward.   \\      



\clearpage
\onecolumn
%SECTION 11: Figures
\section{Figures}

\begin{figure} [htbp]
	\centering	
	\includegraphics[scale=1]{state-diagram.jpg}
	\caption{SLC3 Machine State Diagram	\label{fig:state-diagram}}
\end{figure}


\begin{figure} [htbp]
	\centering
	\includegraphics[scale=0.27]{SLC3_Circuit.png}
	\caption{SLC3 CPU \label{fig:SLC3-Circuit}}
\end{figure}

\begin{figure} [htbp]
	\centering
	\includegraphics[scale=0.4]{Memory_Circuit.png}
	\caption{Memory, MAR, MDR, Mem2IO Configuration\label{fig:Memory-Circuit}}
\end{figure}


%SECTION : Bibliography
%Insert Bibliography if needed

\end{document}
