% % % % % % % % % % % % % % % % % % % % % % % % % % % % % % % % % % % % % % % %
% IEEE Style - Double columns, 11pt font, letterpaper
\documentclass[journal, twocolumn, final,11pt,letterpaper]{IEEEtran}	

% Include Latex Packages
\usepackage{etex}	% This package enables the use of many packages

% % Page styles
\usepackage{setspace}	% line spacing package
\doublespacing			% use double spacing
%\linespread{1.6}		% Use linespread to fine tune line spacing, not recommended


% % Figures
\usepackage{float}		% improves interface for floating objects
\usepackage{subfig}		% enables subfloat
\usepackage{graphicx}	% more image type support
\usepackage{epstopdf}	% automatically convert included eps files to pdf

% % Maths
\usepackage[cmex10]{amsmath}	% Maths
\usepackage{amsfonts,amssymb} 	% maths symbols

% % Tables
\usepackage{booktabs}  % professional-looking tables
\usepackage{multicol} %used for getting multicolumn without page-break
\usepackage{multirow}	% multi-row tables
\usepackage{array}		% define column format of a table

% % Others
\usepackage{caption}	%Customising captions in floating environments
%\usepackage{abstract}
\usepackage{cite}		% cite multiple
\usepackage{fixltx2e}	%added by pilawa, preventing figure* to get ahead of regular figures.
\usepackage{url}		% url display

% %
\hyphenation{op-tical net-works semi-conduc-tor}	% correct bad hyphenation here
\providecommand{\e}[1]{\ensuremath{\times 10^{#1}}}		% use use \e{2} for scientific number expression


% % Optional packages that might be useful
%\usepackage{epsf}		% eps fix
%\usepackage{verbatim}	% verbatim text are not interpreted by the compiler 
%\numberwithin{equation}{section}	% number equation according to section
%\usepackage{xfrac}		% slanted fraction
%\usepackage{pgfplots}	% plot graph
%\usepackage{tikz,pgfplots} % plot graph
%\usepackage{endnotes}	% endnotes


% Title of Document
\title{ECE385 Experiment \#6
	}
\author{
\IEEEauthorblockN{Eric Meyers, Ryan Helsdingen}\\
\IEEEauthorblockA{Section ABG; TAs: Ben Delay, Shuo Liu \\
March 9th, 2016 \\
emeyer7, helsdin2}}
% % % % % % % % % % % % % % % % % % % % % % % % % % % % % % % % % % % % % % % 
\begin{document}
	
%SECTION : Formatting and Title
\maketitle
\singlespacing

%SECTION 1 - Introduction - Eric
\section{Introduction}
The purpose of this lab is to create a very primitive processing unit designed around the Little Computer 3 (LC3) that was explored during previous ECE curriculum. This will be referred to the SLC3 Processing Unit throughout this lab. The SLC3 is a condensed version of the LC3 that allows user interfacing through memory-mapped I/O on board the Altera Cyclone IV SRAM Module, along with switches and LED indicators to show the user the status of the data within registers of the SLC3.  

%SECTION 2 - Description of Circuit - 
\section{Description of Circuit}
The circuit consists of several modules; specifically the high level SLC­3 module, the register file, the datapath, the Instruction Decoder/Sequencer Unit (IDSU), the Arithmetic and Logic Unit (ALU), many 16-bit registers (MAR, MDR, IR, and PC), several multiplexers, and some tristate buffers. The datapath and ISDU (Control) are shown in Figure \ref{fig:SLC3-Circuit}. This Figure directly below this in the same section (Figure \ref{fig:Memory-Circuit}) is the memory interface of the SLC3.\\

All of these modules work together with one another to form the top level SLC3. The SLC3 will perform a total of 

TODO

%SECTION 3 - Purpose of Modules - 
\section{Purpose of Modules}
As stated in the previous section there are many modules that work together in this system to form the top level SLC3. The following modules were created:\\

\normalsize\textbf{16-bit Shift Register} \\
These modules are 16 inputs and 16 outputs with a load\_enable line that determines if the data-in is sent to the data-out. The Instruction Register, Memory Address Register, Memory Data Register, Program Counter Register, and Register File will all be utilizing this module. \\

\normalsize\textbf{Multiplexers} \\
A 16-bit 2-to-1 MUX will be used for the ADDR1MUX and the SR2MUX. A 3-bit 2-to-1 MUX will be used for the DRMUX and the SR1MUX.

A 16-bit 4-to-1 MUX will be used for the ADDR2MUX only.

A 16-bit 3-to-1 MUX will be used for the PCMUX. This will take inputs of the databus, the 16-bit adder output, and the PC+1. \\

\normalsize\textbf{Instruction Sequencer and Decoder } \\
This module will contain the state machine for the entire SLC3. It will have the ability to implement 11 states and cycle through all of them in a cyclic fashion. \\

\normalsize\textbf{NZP} \\
TODO \\

\normalsize\textbf{Datapath} \\
TODO \\

\normalsize\textbf{Arthmetic and Logic Unit} \\
TODO \\

\normalsize\textbf{SEXT/ZEXT} \\
TODO \\

\normalsize\textbf{Register File} \\
TODO \\

\normalsize\textbf{16-bit Adder} \\
TODO \\

\normalsize\textbf{Tristate Buffer} \\
TODO \\

\normalsize\textbf{Tristate} \\
TODO \\

\normalsize\textbf{Mem2IO} \\
TODO \\




%SECTION 4 - State Diagram - 
\section{State Diagram}
RYAN SECTION 

%SECTION 5 - Instruction Sequencer / Decoder - 
\section{Instruction Sequencer / Decoder}
ERIC SECTION 

%SECTION 6 - Schematic/Block Diagram - 
\section{Schematic/Block Diagram}
RYAN SECTION\\

The Schematic / Block Diagrams can be found on Figure \ref{} in "Section XI: Figures".

%SECTION 7 - Pre-Lab Simulation Waveforms -
\section{Pre-Lab Simulation Waveforms}
The Pre-Lab Simulation Waveforms can be found on Figure \ref{} in "Section XI: Figures".

%SECTION 8 - Design Statistics
\section{Design Statistics}
ERIC SECTION

%SECTION 9 - Post Lab -Ryan
\section{Post Lab}
RYAN SECTION - Answer Questions \\

1.) What is MEM2IO used for, i.e. what is its main function? \\

2.) What is the difference between BR and JMP instructions?\\

%SECTION 10 - Conclusion - Ryan
\section{Conclusion}
RYAN SECTION

\clearpage
\onecolumn
%SECTION 11: Figures
\section{Figures}

\begin{figure} [htbp]
	\centering
	\includegraphics[scale=0.27]{SLC3_Circuit.png}
	\caption{SLC3 CPU \label{fig:SLC3-Circuit}}
\end{figure}

\begin{figure} [htbp]
	\centering
	\includegraphics[scale=0.4]{Memory_Circuit.png}
	\caption{Memory, MAR, MDR, Mem2IO Configuration\label{fig:Memory-Circuit}}
\end{figure}


%SECTION : Bibliography
%Insert Bibliography if needed

\end{document}
