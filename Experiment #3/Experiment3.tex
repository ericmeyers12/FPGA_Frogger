% % % % % % % % % % % % % % % % % % % % % % % % % % % % % % % % % % % % % % % %
% IEEE Style - Double columns, 11pt font, letterpaper
\documentclass[journal, twocolumn, final,11pt,letterpaper]{IEEEtran}	

% Include Latex Packages
\usepackage{etex}	% This package enables the use of many packages

% % Page styles
\usepackage{setspace}	% line spacing package
\doublespacing			% use double spacing
%\linespread{1.6}		% Use linespread to fine tune line spacing, not recommended


% % Figures
\usepackage{float}		% improves interface for floating objects
\usepackage{subfig}		% enables subfloat
\usepackage{graphicx}	% more image type support
\usepackage{epstopdf}	% automatically convert included eps files to pdf

% % Maths
\usepackage[cmex10]{amsmath}	% Maths
\usepackage{amsfonts,amssymb} 	% maths symbols

% % Tables
\usepackage{booktabs}  % professional-looking tables
\usepackage{multicol} %used for getting multicolumn without page-break
\usepackage{multirow}	% multi-row tables
\usepackage{array}		% define column format of a table

% % Others
\usepackage{caption}	%Customising captions in floating environments
%\usepackage{abstract}
\usepackage{cite}		% cite multiple
\usepackage{fixltx2e}	%added by pilawa, preventing figure* to get ahead of regular figures.
\usepackage{url}		% url display

% %
\hyphenation{op-tical net-works semi-conduc-tor}	% correct bad hyphenation here
\providecommand{\e}[1]{\ensuremath{\times 10^{#1}}}		% use use \e{2} for scientific number expression


% % Optional packages that might be useful
%\usepackage{epsf}		% eps fix
%\usepackage{verbatim}	% verbatim text are not interpreted by the compiler 
%\numberwithin{equation}{section}	% number equation according to section
%\usepackage{xfrac}		% slanted fraction
%\usepackage{pgfplots}	% plot graph
%\usepackage{tikz,pgfplots} % plot graph
%\usepackage{endnotes}	% endnotes


% Title of Document
\title{ECE385 Experiment \#3
	}
\author{
\IEEEauthorblockN{Eric Meyers, Ryan Helsdingen}\\
\IEEEauthorblockA{Section ABG; TAs: Ben Delay, Shuo Liu \\
February 10th, 2016 \\
emeyer7, helsdin2}}
% % % % % % % % % % % % % % % % % % % % % % % % % % % % % % % % % % % % % % % 
\begin{document}
	
%SECTION : Formatting and Title
\maketitle
\singlespacing

%SECTION 1 - Introduction - Eric
\section{Introduction}
\IEEEPARstart{T}{he} purpose of this lab is to design and construct a four-bit serial logic processor that performs a total of eight logical operations in a bit-wise fashion. There are two four-bit words stored in two shift registers, A and B. The operator may store any data of their choosing in these two shift registers by using the load/data-in switches and then execute any logical bitwise operation on these two words by using the function-select switches along with the routing switches.\\
\vspace{-5mm}

%SECTION 2 - Pre-Lab  - Eric
\section{Pre-Lab}
Part A) Describe the simplest (two-input one-output) circuit that can optionally invert a signal (i.e., one input determines if the output is equal to the other input or equal to the other input inverted). Sketch your circuit. \\

Answer: The logic needed is shown in the following truth table:
\begin{center}
	\begin{tabular}{ll|l}
		A & B & Y \\ \hline
		0 & 0 & 0 \\
		0 & 1 & 1 \\
		1 & 0 & 1 \\
		1 & 1 & 0 \\
	\end{tabular}
\end{center}

Taking B to be the "select" line in this example, when the input (A) is driven high and our select line (B) is low, the output will be high. The reverse case is true when the input (A) is low and the select line (B) is high, the output will also be high. Therefore, depending on the select line, the input will be inverted. This can be implemented with a single XOR gate as shown below. \\
\begin{center}
	\begin{circuitikz} \draw
	(2,1) node[xor port] (myxor) {}
	(myxor.in 1) node[left=.5cm](a) {A}
	(myxor.in 2) node[left = .5cm](b) {B}
	(myxor.out)node[right = .5cm](y){Y};
	\end{circuitikz}
\end{center}
\vspace{5mm}

Part B) Explain how a modular design such as that presented above improves testability and cuts down development time. Propose an approach that could be used to troubleshoot the modular circuit above if it appeared to be completing the computation cycle correctly but was not giving the correct output. (Be specific.)\\

Answer: A modular design allows the operator to unit-test each individual module. Once each module is successfully tested, the entire system can be integrated into a whole-system and tested. A modular design ideally will cut down the debugging time needed on the system as a whole.  


%SECTION 3 - Description of Circuit - Eric
\section{Description of Circuit}
The logic processor was designed to contain two storage registers, and allow the operator to specify the data loaded into each register (A and B), as well as the operation performed between the two register, where the output is stored and 
The team designed a total of four modular sub-systems in this lab. The following sub-systems were designed and constructed:\\
\begin{enumerate}
	\item Data Loader Unit
	\item Computation Unit
	\item Function-Select and Output Router Unit
	\item Control Logic Finite State Machine (FSM) Unit
\end{enumerate}
\vspace{2mm}

Each sub-system was tested individually and system integration tests were performed at the end upon successfully unit-testing each component. This way the team could eliminate errors and debug with efficiency.\\

A total of eight operations can be performed on this unit. AND, OR, XOR, NAND, NOR, and NXOR are the six primary logical operations that can be performed, along with the last two operations simply being loading the specified registers with all 1s or 0s. \\

The Data Loader consists of a total of six switches. Two switches for loading either register A or B, and four switches to specify the contents to store in the particular register.\\

The Computation Unit performed the selected operation specified through the three function-select switches. The output of the operation is funnelled into the proper register as given by the operator. \\

Finally, the Control Logic/Finite State Machine Unit is a sequential circuit that essentially determines the correct  number of cycles to perform a full logical operation. It relies on one input (execute switch) and has one output that determines if the registers should be shifting right, doing nothing, or loading. \\

In the end, the functionality of the circuit is as follows: when a user selects a four-bit word and loads it into the register of their choosing, they must then select the logical operation they would like to perform, select where they would like the output of this logical operation to go via the routing selection, and flip the execute switch. This will begin shifting the outputs of the shift register and perform bitwise logical operations of the requestion function. The machine will halt after exactly four operations and the register requested through the routing switches will contain the output of the logical operation. \\

%SECTION 4 - State Diagram - Eric
\section{State Diagram}
The FSM created in this lab is a Mealy State Machine and this was chosen due to the simplicity of the relationship between the input and output on the state diagram. (???????)

\begin{figure} [H]
	\centering
	\includegraphics[scale=0.35]{FSM.png}
	\label{fig:mealy-fsm}
	\caption{Mealy FSM}
\end{figure}    

%SECTION 5 - Design - Eric
\section{Design}
The first component designed in this lab was the FSM Sequential Logic Circuit. This is because the output of this state machine controls the actions on the shift register (i.e. this output controls the S1 and S0 "control bits" , which determines whether the registers are shifting right, parallel loading, or doing nothing). This is important because it determines whether the machine is in a load/execute/do nothing state. These proper inputs to the control bits on the shift registers will be examined further on in this section.\\

The truth table for the FSM was given from the documentation in the Experiment 3 PDF. This is shown below in figure.

\begin{figure} [H]
	\centering
	\includegraphics[scale=0.33]{FSM_Truth_Table.png}
	\label{fig:fsm-truth-table}
	\caption{FSM Truth Table}
\end{figure}       

Next, given this truth table, the team constructed a K-Map for each of the output bits - S, Q , C1 and C0. Since the only output is "S", and the remaining bits are only internal to the state machine, there must be three D-flip-flops to store the state of these internal bits. The K-Maps are shown along with the cooresponding S.O.P resultant equations.

\begin{figure} [H]
	\centering
	\begin{Karnaugh}
		\contingut{0,x,x,x,0,1,1,1,1,x,x,x,0,1,1,1}
		\implicant{1}{11}{gray}
		\implicant{3}{10}{gray}
	\end{Karnaugh}
	\vspace{-5mm}
	\caption*{FSM Sequential Circuit Output (S)}
\end{figure}
\vspace{-5mm}
\[ S = C_1 \lor C_0 \lor \lnot E \land Q\]\\


\begin{figure} [H]
	\centering
	\begin{Karnaugh}
		\contingut{0,x,x,x,0,1,1,1,1,x,x,x,1,1,1,1}
		\implicant{1}{11}{gray}
		\implicant{3}{10}{gray}
		\implicant{12}{10}{gray}
	\end{Karnaugh}
	\vspace{-5mm}
	\caption*{State Internal Bit (Q)}
\end{figure}
\vspace{-5mm}
\[ Q = C_1 \lor C_0 \lor E\]


\begin{figure} [H]
	\centering
	\begin{Karnaugh}
		\contingut{0,x,x,x,0,1,1,0,0,x,x,x,0,1,1,0}
		\implicant{1}{9}{gray}
		\implicant{2}{10}{gray}
	\end{Karnaugh}
	\vspace{-5mm}
	\caption*{Counter Internal Bit (C1)}
\end{figure}
\vspace{-5mm}
\[ C_1 = C_1 \oplus C_0 \]

\begin{figure} [H]
	\centering
	\begin{Karnaugh}
		\contingut{0,x,x,x,0,0,1,0,1,x,x,x,0,0,1,0}
		\implicant{2}{10}{gray}
		\implicant{8}{10}{gray}
	\end{Karnaugh}
	\vspace{-5mm}
	\caption*{Counter Internal Bit (C0}
\end{figure}
\vspace{-5mm}
\[ Q = (C_1 \land \lnot C_0) \lor (\lnot E \land Q)\]

Once these logic functions were derived, an AND-OR circuit was designed and then converted into a NAND-NOR circuit. This NAND-NOR circuit was optimized to use the least number of chips possible and as a result, only used a total of 5 TTL chips.\\ 

The next modular component that was designed was the computational and router unit. This was the circuit that performed a particular logical operation. The three primary functions (AND, OR, XOR) and a high signal is fed into a 4-to-1 MUX with the select lines connected to the F1 and F0 switches. This output of the 4-to-1 MUX is then fed into a comparator. The second line of this comparator is connected to the F2 switch (negated). This effectively acts as an XOR gate. This is so that we do not have to use extra logic gates on the second set of logical operations (NAND, NOR, NXOR, 0000), and instead use an optional inverter, which was demonstrated in the pre-lab.  \\

This output is then fed to the proper inputs on the two 4-to-1 MUXs connected to the shift registers. The R1 and R0 control the routing selection, and the two MUXs on the top must be fed the right values in order to route the output correctly and perform the proper action. This output of the MUX then is fed into the left serial input on the shift register. \\

Next the shift register control bits needed to be determined to either be shifting right parallel loading or doing nothing. S1 and S0 must both be high if a parallel load is being performed on the shift register, so this means the LoadA switch can be wired directly to the A shift register S1 control bit, and the LoadB switch can be wired directly to the B shift register S1 control bit. The S0 must come from the Control Logic Circuit (the "S" output), and be fed into some combinational logic that is derived below:

\begin{center}
	\begin{tabular}{lll|ll|ll}
		S & LA & LB & S1A & S0A & S1B & S0B\\ \hline
		0 & 0 & 0 & 0 & 0 & 0 & 0 \\
		0 & 0 & 1 & 0 & 0 & 1 & 1 \\
		0 & 1 & 0 & 1 & 1 & 0 & 0 \\
		0 & 1 & 1 & 1 & 1 & 1 & 1 \\
		1 & 0 & 0 & 0 & 1 & 0 & 1 \\
		1 & 0 & 1 & 0 & 0 & 1 & 1 \\
		1 & 1 & 0 & 1 & 1 & 0 & 0 \\
		1 & 1 & 1 & 1 & 1 & 1 & 1 \\
	\end{tabular}
\end{center}

This is referencing the fact that the control bits on each Register A and Register B are as follows: 

\begin{center}
	\begin{tabular}{ll|l}
		S1 & S0 & Action \\ \hline
		0 & 0 & Do Nothing  \\
		0 & 1 & Shift Right \\
		1 & 0 & Shift Left \\
		1 & 1 & Parallel Load\\
	\end{tabular}
\end{center}

Next, from the above first truth table K-Maps can be formed, this is shown below with the cooresponding S.O.P functions along side them.
\begin{figure} [H]
	\centering
	\begin{Karnaughvuit}
		\minterms{2,3,6,7}
		\maxterms{0,1,4,5}
		\implicant{3}{6}{gray}
	\end{Karnaughvuit}
	\vspace{-15mm}
	\caption*{}
\end{figure}
\vspace{-5mm}
\[ S_1 : A=  LoadA\] 


\begin{figure} [H]
	\centering
	\begin{Karnaughvuit}
		\minterms{2,4,3,6,7}
		\maxterms{0,1,5}
		\implicant{3}{6}{gray}
		\implicantcostats{4}{6}{gray}
	\end{Karnaughvuit}
	\vspace{-5mm}
	\caption*{}
\end{figure}
\vspace{-15mm}
\[ S_0:A=  \lnot LoadA \lor (\lnot LoadB \land S)\] 


\begin{figure} [h]
	\centering
	\begin{Karnaughvuit}
		\minterms{1,3,5,7}
		\maxterms{0,4,2,6}
		\implicant{1}{7}{gray}
	\end{Karnaughvuit}
	\vspace{-15mm}
	\caption*{}
\end{figure}
\vspace{-5mm}
\[ S_1:B=  LoadB\] 


\begin{figure} [h]
	\centering
	\begin{Karnaughvuit}
		\minterms{1,3,4,5,7}
		\maxterms{0,2,6}
		\implicant{1}{7}{gray}
		\implicant{4}{5}{gray}
	\end{Karnaughvuit}
	\vspace{-15mm}
	\caption*{}
\end{figure}
\vspace{-5mm}
\[ S_0:B=  \lnot LoadB \lor (\lnot LoadA \land S)\] 

Once these equations were formed, an AND-OR circuit was designed, then from this a NAND-NAND circuit containing only four logic gates was created. This serves as the control of our shift registers and will feed to the proper place upon constructing.\\ 

All in all, the total number of chips used was 18 TTL chips. This chip count could have been reduced if better optimization was performed, however, since the team possessed the required chips, none was needed. The chips used were as follows:
\begin{itemize}
	\item Control Logic Unit
	\begin{itemize}
		\item 74LS00 - 2-input NAND (2)
		\item 74LS02 - 2-input NOR
		\item74LS04 - Hex Inverter
		\item 74LS27 - 3-input NOR
		\item 74LS107 - J-K Flip Flop (2)
	\end{itemize}
	\item Computation and Router Unit
	\begin{itemize}
		\item 74LS00 - 2-input NAND
		\item 74LS85 Comparator
		\item 74LS153 4-to-1 Multiplexer
		\item 74LS194 - Universal Shift Register (2)
	\end{itemize}
	\item Loader Unit
	\begin{itemize}
		\item 74LS02 - 2-input NOR (3)
		\item 74LS04 - Hex Inverter 
	\end{itemize}
\end{itemize}
%SECTION 6 - Block Diagram - Ryan
\section{Block Diagram} 
RYAN SECTION

%Please refer to Figure \_\_\_\_ in "Section XI: Figures" of this document to view the Block Diagram created in this lab.
 
%SECTION 7 - Circuit/Logic Diagrams - Eric
\section{Circuit/Logic Diagrams}
As stated before, to allow for easier  construction and debugging, this system was broken down into modular components. All circuit diagrams are shown in "Section XI: Figures" of this document.\\
 
The FSM Control Logic is shown in Figure \ref{fig:control-logic-diagram}. The Loader is shown in Figure \ref{fig:loader-logic-diagram} and the Computational/Router Unit is shown in Figure \ref{fig:computational-router-logic-diagram}.   
 
%SECTION 8 - Component Layout Sheet - Eric 
\section{Component Layout Sheet}
Please refer to Figure \ref{fig:component-layout} in "Section XI: Figures" of this document to view the Component Layout Sheet created in this lab.

%SECTION 9 - Documentation - Ryan
\section{Documentation from Experiment}
RYAN SECTION


%SECTION 10 - Conclusion - Ryan
\section{Conclusion}
RYAN SECTION

\clearpage
\onecolumn
%SECTION 11: Figures
\section{Figures}

\begin{figure} [H]
	\centering
	\includegraphics[scale=0.55]{Control_Logic.png}
	\caption{Control Logic Diagram\label{fig:control-logic-diagram}}
\end{figure}

\begin{figure} [H]
	\centering
	\includegraphics[scale=0.6]{Loader_Logic.png}
	\caption{Loader Logic Diagram\label{fig:loader-logic-diagram}}
\end{figure}

\begin{figure} [H]
 	\centering
 	\includegraphics[scale=0.55]{Computational_Router_Logic.png}
 	\caption{Computational/Router Logic Diagram\label{fig:computational-router-logic-diagram}}
\end{figure}

\begin{figure} [H]
	\centering
	\includegraphics[scale=0.55]{Component_Layout.png}
	\caption{Component Layout Diagram\label{fig:component-layout}}
\end{figure}



%SECTION : Bibliography
%Insert Bibliography if needed

\end{document}